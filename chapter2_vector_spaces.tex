\documentclass{article}
\usepackage{array}
\usepackage{amssymb}
\usepackage{amsmath}
\usepackage{enumitem}
\usepackage{hyperref}
\hypersetup{
    bookmarksnumbered=true,
    pdfpagemode=UseOutlines,
}

\usepackage{array}

\setlist[enumerate]{font=\bfseries}

% The first argument is the number of columns before the divider,
% and the second argument is the number of columns before the divider,
\newenvironment{abmatrix}[2]
{
    \left[
        \begin{array} {@{} *{#1}{c} | *{#2}c @{}}
}
{
        \end{array}
    \right]
}


% The argument is the number of variables in the system.
\newcolumntype{B}{>{{}}c<{{}}}
\newenvironment{system}[1]
{
    \begin{cases}
        \setlength{\arraycolsep}{0pt}
        \begin{array} {r @{}*{#1}{B r}}
}
{
        \end{array}
    \end{cases}
}

\DeclareMathOperator{\rank}{rank}
\DeclareMathOperator{\nullity}{nullity}
\DeclareMathOperator{\Span}{span}
\DeclareMathOperator{\tr}{tr}


\begin{document}

\setcounter{section}{1}
\section{Vector Spaces}

\subsection{Vector Spaces}
\begin{enumerate}[listparindent=\parindent]
\item[1.] If \(F\) is a field, verify that \(F^n\) (as defined in Example 1) is a vector space over the field \(F\).

Let \(\alpha = (x_1, x_2, \dots, x_n)\), \(\beta = (y_1, y_2, \dots, y_n)\), and \(\gamma = (z_1, z_2, \dots, z_n)\).
First we prove the properties of vector addition, using the definition of addition over a field:
    \begin{enumerate}[listparindent=\parindent]
        \item[(a)] Addition is commutative.
        \begin{gather*}
            \alpha + \beta \\
            = (x_1 + y_1, x_2 + y_2, \dots, x_n + y_n) \\
            = (y_1 + x_1, y_2 + x_2, \dots, y_n + x_n) \\
            = \beta + \alpha
        \end{gather*}

        \item[(b)] Addition is associative.
        \begin{gather*}
            (\alpha + \beta) + \gamma \\
            = ((x_1 + y_1) + z_1, (x_2 + y_2) + z_2, \dots, (x_n + y_n) + z_n) \\
            = (x_1 + (y_1 + z_1), x_2 + (y_2 + z_2), \dots, x_n + (y_n + z_n)) \\
            = \alpha + (\beta + \gamma)
        \end{gather*}

        \item[(c)] There is a unique vector 0 such that \(\alpha + 0 = \alpha\) for all \(\alpha\) in V.

            Since \(F\) is a field, there exists a unique element 0 in \(F\) such that \(x + 0 = x\) for all \(x\) in \(F\),
            so there is a unique zero vector which is a \(n\)-tuple containing all 0s:
            \[(x_1 + 0, \dots, x_n + 0) = (x_1, \dots, x_n) + (0, \dots, 0) = \alpha + 0\]

        \item[(d)] For each vector \(\alpha\) in \(V\) there is a unique vector \(-\alpha\) in \(V\) such that \(\alpha + (-\alpha) = 0\).

            Since \(F\) is a field, there exists a unique element \(-x\) in \(F\) such that \(x + (-x) = x\) for all \(x\) in \(F\). Therefore,
            \begin{gather*}
                (x_1 + -(x_1), \dots, x_n + -(x_n)) \\
                = (x_1, \dots, x_n) + (-x_1, \dots, -x_n) \\
                = \alpha + -\alpha
            \end{gather*}
    \end{enumerate}

Then we prove the properties of scalar multiplication:
    \begin{enumerate}[listparindent=\parindent]
        \item[(a)] \(1\alpha = \alpha\) for every \(\alpha \in V\).
        \[1\alpha = (1x_1, \dots, 1x_n) = (x_1, \dots, x_n) = \alpha\]

        \item[(b)] \((c_1c_2)\alpha = c_1(c_2\alpha)\).
        \begin{gather*}
            (c_1c_2)\alpha = ((c_1c_2)x_1, \dots, (c_1c_2)x_n) \\
            = (c_1(c_2x_1), \dots, c_1(c_wx_n)) \\
            = c_1(c_2\alpha)
        \end{gather*}

        \item[(c)] \(c(\alpha + \beta) = c\alpha + c\beta\)
        \begin{gather*}
            c(\alpha + \beta) = c(x_1 + y_1, \dots, x_n + y_n) \\
            = (c(x_1 + y_1), \dots, c(x_n + y_n)) \\
            = (cx_1 + cy_1, \dots, cx_n + cy_n) \\
            = c\alpha + c\beta
        \end{gather*}

        \item[(d)] \((c_1 + c_2)\alpha = c_1\alpha + c_2\alpha\)
        \begin{gather*}
            (c_1 + c_2)\alpha = ((c_1 + c_2)x_1, \dots, (c_1 + c_2)x_n) \\
            = (c_1x_1 + c_2x_1, \dots, c_1x_n + c_2x_n) \\
            = c_1\alpha + c_2\alpha
        \end{gather*}
    \end{enumerate}

As it satisfies all the properties of a vector space, \(R^n\) is a vector space over \(F\).

\item[2.] If \(V\) is a vector space over the field \(F\), verify that
    \[(\alpha_1 + \alpha_2) + (\alpha_3 + \alpha_4) = [\alpha_2 + (\alpha_3 + \alpha_1)] + \alpha_4\]
    for all vectors \(\alpha_1, \alpha_2, \alpha_3, \alpha_4 \in V\).
    \begin{gather*}
        (\alpha_1 + \alpha_2) + (\alpha_3 + \alpha_4) \\
        = [(\alpha_2 + \alpha_1) + \alpha_3] + \alpha_4 \\
        = [\alpha_2 + (\alpha_1 + \alpha_3)] + \alpha_4 \\
        = [\alpha_2 + (\alpha_3 + \alpha_1)] + \alpha_4
    \end{gather*}

\item[3.] If \(C\) is the field of complex numbers, which vectors in \(C^3\) are linear combinations of
    \((1, 0, -1), (0, 1, 1)\) and \((1, 1, 1)\)?

Let \((y_1, y_2, y_3)\) be a vector formed by linear combinations of the three given vectors.
Then for some scalars \(x_1, x_2, x_3 \in C\),
\[
    \begin{bmatrix}
        1 & 0 & 1 \\
        0 & 1 & 1 \\
        -1 & 1 & 1 \\
    \end{bmatrix}
    \begin{bmatrix}
        x_1 \\ x_2 \\ x_3
    \end{bmatrix}
    =
    \begin{bmatrix}
        y_1 \\ y_2 \\ y_3
    \end{bmatrix}
\]
or in short, \(AX = Y\). Elementary row operations can show that \(A\) is row-equivalent to \(I\),
and therefore \(A\) is invertible. We can then solve the system; \(X = A^{-1}Y\).
This means any vector \(Y\) over \(C\) can be written as a linear combination of the three given vectors.

\item[4.] Let \(V\) be the set of all pairs \((x, y)\) of real numbers, and let \(F\) be the field of real numbers.
    Define

    \[(x, y) + (x_1, y_1) = (x + x_1, y + y_1)\]
    \[c(x, y) = (cx, y).\]
    Is \(V\), with these operations, a vector space over the field of real numbers?

\(V\) is not a vector space because there is no zero scalar \(0\) such that \(0(x, y) = 0\) for all \(y \in V\);
\(0(x, y) = (0x, y) \neq (0, 0)\) unless \(y = 0\).

\item[6.] Let \(V\) be the set of all complex-valued function \(f\) on the real line such that (for all \(t\) in \(\mathbb{R}\))
    \[f(-t) = \overline{f(t)}.\]
    The bar denotes complex conjugation. Show that \(V\), with operations
    \begin{gather*}
        (f + g)(t) = f(t) + g(t) \\
        (cf)(t) = cf(t)
    \end{gather*}
    is a vector space over the field of \textit{real} numbers. Give an example of a function in \(V\) which is not real-valued.

    First, for an example of a not real-valued function in \(V\), let \(f(t) = ti\). Then \(f(-t) = -ti = \overline{ti} = \overline{f(t)}\).

    \(V\) is closed under addition and multiplication. Suppose \(c \in \mathbb {R}\) and \(f, g \in V\). Then,
    \begin{gather*}
        (f + g)(-t) \\
        = f(-t) + g(-t) \\
        = \overline{f(t)} + \overline{g(t)} \\
        = \overline{f(t) + g(t)} \\
        = \overline{(f + g)(t)} \\
        (f + g)(t) \in V
    \end{gather*}
    \begin{gather*}
        (cf)(-t) \\
        = cf(-t) \\
        = c\overline{f(t)} \\
        = \overline{cf(t)}
        = \overline{(cf)(t)} \\
        (cf)(t) \in V
    \end{gather*}

    Commutativity and associativity directly follow from properties of addition and multiplication in \(\mathbb{C}\).
    \[(f + g)(t) = f(t) + g(t) = g(t) + f(t) = (g + f)(t)\]
    \[((f + g) + h)(t) = (f(t) + g(t)) + h(t) = \\ f(t) + (g(t) + h(t)) = (f + (g + h))(t)\]

    The zero vector is the function \(g(t) = 0\); then \((f + g)(t) = f(t) + g(t) = f(t) + 0 = f(t)\).
    The uniqueness follows from the uniqueness of the additive identity in \(\mathbb{C}\).
    \(g\) is also in \(V\) as \(g(-t) = 0 = \overline{0} = \overline{g(t)}\).

    For each vector \(f(t)\), there exists a unique vector \(-f(t)\) such that \((f + (-f))(t) = 0\):
    \[(f + (-f))(t) = f(t) + -f(t) = f(t) - f(t) = 0\]
    It is in \(V\), as \(-f(-t) = -\overline{f(t)} = \overline{-f(t)}\).

    Scalar multiplication is distributive over vector addition. Suppose \(c\) is a real number and \(f, g \in V\). Then,
    \begin{gather*}
        c(f + g)(t) \\
        = c(f(t) + g(t)) \\
        = cf(t) + cg(t) \\
        = (cf + cg)(t) \\
        = (cf)(t) + (cg)(t)
    \end{gather*}
    as multiplication is distributive over addition in \(\mathbb{C}\).

    Finally, suppose \(c_1\) and \(c_2\) are real numbers and \(f \in V\). Then
    \begin{gather*}
        ((c_1 + c_2)f)(t) \\
        = (c_1 + c_2)f(t) \\
        = c_1f(t) + c_2f(t) \\
        = (c_1f + c_2f)(t)
    \end{gather*}
\end{enumerate}

\subsection{Vector Spaces}
\begin{enumerate}[listparindent=\parindent]
\item[1.] Which of the following sets of vectors \(\alpha = (\alpha_1, \dots, \alpha_n)\)
    in \(\mathbb{R}^n\) are subspaces for \(\mathbb{R}^n (n \geq 3)\)?

    For each set below, denote the given set \(S\) and suppose \(\alpha, \beta \in S\) and \(c \in \mathbb{R}\).
    \begin{enumerate}[listparindent=\parindent]
        \item[(a)] all \(\alpha\) such that \(\alpha_1 \geq 0\);

            \(S\) is not a subspace, as \(-\alpha \notin S\) for all \(\alpha \neq 0\).

        \item[(b)] all \(\alpha\) such that \(\alpha_1 + 2\alpha_2 = \alpha_3\);
            Suppose \(\gamma = c\alpha + \beta = (\gamma_1, \gamma_2, \gamma_3, \dots)\);
            \begin{gather*}
                \gamma_1 + 2\gamma_2 \\
                = c(\alpha_1 + \beta_1) + 2c(\alpha_2 + \beta_2) \\
                = c[(\alpha_1 + 2\alpha_2) + (\beta_1 + 2\beta_2)] \\
                = c(\alpha_3 + \beta_3) \\
                = \gamma_3
            \end{gather*}
            Therefore \(\gamma \in S\) and \(S\) is a subspace by Theorem 1.

        \item[(c)] all \(\alpha\) such that \(\alpha_2 = \alpha_1^2\);

            Suppose \(\gamma = \alpha + \beta\);
            then \(\gamma_2 = \alpha_2 + \beta_2 = \alpha_1^2 + \beta_1^2 \neq (\alpha_1 + \beta_1)^2 = \gamma_1^2\).
            \(\gamma \notin S\), therefore \(S\) is not a subspace.

        \item[(d)] all \(\alpha\) such that \(\alpha_1\alpha_2 = 0\).

            Suppose \(\gamma = c\alpha + \beta = (\gamma_1, \gamma_2, \dots)\);
            \begin{gather*}
                \gamma_1\gamma_2 = c^2(\alpha_1 + \beta_1)(\alpha_2 + \beta_2) \\
                = c^2(\alpha_1\alpha_2 + \alpha_2\beta_1 + \alpha_1\beta_2 + \beta_1\beta_2) \\
                = c^2(\alpha_2\beta_1 + \alpha_1\beta_2)
            \end{gather*}
            At least one of \(\alpha_1\) and \(\alpha_2\) must be 0, and at least one of \(\beta_1\) and \(\beta_2\) must be 0.
            However, this does not guarantee that \(\alpha_2\beta_1 + \alpha_1\beta_2 = 0\), so \(\gamma \notin S\) for some \(\alpha\) and \(\beta\),
            Therefore \(S\) is not a subspace.

        \item[(e)] all \(\alpha\) such that \(\alpha_2\) is rational.

            Suppose \(\gamma = \sqrt{2}\alpha\). Then \(\gamma_2 = \sqrt{2}\alpha_2\) is not rational
            and \(\gamma \notin S\), so \(S\) is not a subspace.

    \end{enumerate}
\item[2.] Let \(V\) be the (real) vector space of all functions \(f\) from \(R\) to \(R\).
    Which of the following are subspaces of \(V\)?

    For each set below, denote the given set \(W\) and suppose \(f, g \in W\) and \(c \in \mathbb{R}\).
    \begin{enumerate}[listparindent=\parindent]
        \item[(a)] all \(f\) such that \(f(x^2) = f(x)^2\);

            Suppose \(f(x) = x\) and \(g(x) = x^2\). \(f, g \in W\) because
            \(f(x^2) = x^2 = f(x)^2\) and \(g(x^2) = x^4 = (x^2)^2 = g(x)^2\).

            Then \(h(x) = (f + g)(x) = x + x^2\), but \(h(x^2) = x^2 + x^4\) and
            \(h(x)^2 = (x + x^2)^2 = x^2 + 2x^3 + x^4\). Therefore \(h(x) \notin S\)
            and this set is not a subspace.

        \item[(b)] all \(f\) such that \(f(0) = f(1)\).

            \(h(0) = cf(0) + g(0)\) and \(h(1) = cf(1) + g(1)\).
            \(f(0) = f(1)\) and \(g(0) + g(1)\), so it must be the case that \(h(0) = h(1)\),
            and \(h \in W\). By Theorem 1, \(W\) is a subspace of \(V\).

        \item[(c)] all \(f\) such that \(f(3) = 1 + f(-5)\).

            Suppose \(h = f + g\).
            It can be shown that \(h(3) \neq 1 + h(-5)\);

            \begin{gather*}
                h(3) = f(3) + g(3) \\
                = 1 + f(-5) + 1 + g(-5) \\
                = 2 + f(-5) + g(-5) \\
                = 2 + h(-5) \\
                \neq 1 + h(-5)
            \end{gather*}

            As \(f + g \notin W\), \(W\) is not a subspace of \(V\).

        \item[(d)] all \(f\) such that \(f(-1) = 0\);

            Suppose \(h = cf + g\). Then \(h \in W\);
            \begin{gather*}
                h(-1) = cf(-1) + g(-1) \\
                = c(0) + 0 \\
                = 0
            \end{gather*}
            By Theorem 1, \(W\) is a subspace of \(V\).

        \item[(e)] all \(f\) which are continuous.

            Suppose \(h = cf + g\). It must be that \(h \in W\)
            since the sum of two continuous functions is continous,
            and if \(f\) is a continuous function then \(cf\) is a continuous function for all \(c \in \mathbb{R}\).
            By Theorem 1, \(W\) is a subspace of \(V\).
    \end{enumerate}

\item[3.] Is the vector (3, -1, 0, -1) in the subspace of \(\mathbb{R}^4\) spanned by the vectors
    (2, -1, 3, 2), (-1, 1, 1, -3), and (1, 1, 9, -5)?

    The subspace spanned by a set of vectors is the set of all linear combinations of the vectors.
    To check if the given vector is a linear combination of the three other vectors, we solve for \(X\) in
    \[
        \begin{bmatrix}
            2 & -1 & 1 \\
            -1 & 1 & 1 \\
            3 & 1 & 9 \\
            2 & -3 & -5
        \end{bmatrix}
        X
        =
        \begin{bmatrix}
            3 \\ -1 \\ 0 \\ -1
        \end{bmatrix}
    \]
    and see that there is no solution for the equation.
    As \((3, -1, 0, -1)\) is cannot be written as a linear combination of the other vectors,
    it is not in the subspace spanned by the three vectors.

\item[4.] Let \(W\) be the set of all \((x_1, x_2, x_3, x_4, x_5)\) in \(\mathbb{R}^5\) which satisfy
    \[
        \begin{system}{5}
            2x_1 &-& x_2 &+& \frac{4}{3}x_3 &-& x_4 && &=& 0 \\
            x_1 && &+& \frac{2}{3}x_3 && &-& x_5 &=& 0 \\
            9x_1 &-& 3x_2 &+& 6x_3 &-& 3x_4 &-& 3x_5 &=& 0 \\
        \end{system}
    \]
    Find a finite set of vectors which spans \(W\).

\[
    \begin{bmatrix}
        2 & -1 & \frac{4}{3} & -1 & 0 \\
        1 & 0 & \frac{2}{3} & 0 & -1 \\
        9 & -3 & 6 & -3 & -3
    \end{bmatrix}
\]

The reduced row-echelon matrix is
\[
    \begin{bmatrix}
        1 & 0 & \frac{2}{3} & 0 & -1 \\
        0 & 1 & 0 & 1 & -2 \\
        0 & 0 & 0 & 0 & 0
    \end{bmatrix}
\]

Therefore, all vectors in \(W\) are in the form
\[ (-\frac{2}{3}x_3 + x_5, -x_4 + 2x_5, x_3, x_4, x_5). \]

\{(1, 2, 0, 0, 1), (0, -1, 0, 1, 0), (-2, 0, 3, 0, 0)\} 
is an example of set of vectors that span \(W\).

\item[5.] Let \(F\) be a field and let \(n\) be a positive integer \((n \geq 2)\).
    Let \(V\) be the vector space of all \(n \times n\) matrices over \(F\).
    Which of the following sets of matrices \(A\) in \(V\) are subspaces of \(V\)?
    \begin{enumerate}[listparindent=\parindent]
        \item[(a)] all invertible \(A\);

            The set of all invertible matrices is not a subspace of \(V\)
            because the zero matrix, which is the zero vector in \(V\), is not invertible.

        \item[(b)] all non-invertible \(A\);

            The set of all non-invertible matrices is not a subspace of \(V\)
            since two non-invertible matrices can sum to an invertible matrix;
            \[
                \begin{bmatrix}
                    1 & 0 \\
                    0 & 0 \\
                \end{bmatrix}
                +
                \begin{bmatrix}
                    0 & 0 \\
                    0 & 1 \\
                \end{bmatrix}
                =
                \begin{bmatrix}
                    1 & 0 \\
                    0 & 1 \\
                \end{bmatrix}
            \]

        \item[(c)] all \(A\) such that \(AB = BA\), where \(B\) is some fixed matrix in \(V\);

            Let the set of all matrices \(A\) such that \(AB = BA\) be \(W\).
            Suppose \(A_1\) and \(A_2\) are matrices in \(W\). Then any matrix in the form
            \(cA_1 + A_2\) where \(c \in F\) is in \(W\);
            \begin{gather*}
                (cA_1 + A_2)B \\
                = cA_1B + A_2B \\
                = cBA_1 + BA_2 \\
                = B(cA_1) + BA_2 \\
                = B(cA_1 + A_2) \\
            \end{gather*}

            By Theorem 1, the set of all \(A\) such that \(AB = BA\) is a subspace of \(V\).

        \item[(d)] all \(A\) such that \(A^2 = A\).

            Suppose \(I\) is the \(2 \times 2\) identity matrix;
            \[
                I + I
                =
                \begin{bmatrix}
                    2 & 0 \\
                    0 & 2 \\
                \end{bmatrix}
            \]

            Clearly, \(I^2 = I\). However
            \[
                \begin{bmatrix}
                    2 & 0 \\
                    0 & 2 \\
                \end{bmatrix}^2
                =
                \begin{bmatrix}
                    4 & 0 \\
                    0 & 4 \\
                \end{bmatrix}
            \]

            As it is not closed under addition, the set is not a subspace of \(V\).
    \end{enumerate}

\item[6.]
    \begin{enumerate}[listparindent=\parindent]
        \item[(a)] Prove that the only subspaces of \(\mathbb{R}^1\) are \(\mathbb{R}^1\) and the zero subspace.

            First, any space trivially has the zero subspace as its subspace.
            Suppose \(S\) be a subspace of \(\mathbb{R}^1\) that is not the zero subspace.
            \(S\) is the set of all linear combinations \(cv\) for some non-zero vector \(v = (v_1)\) in \(S\) and some scalar \(c\).
            Then vector \(u = (u_1)\) in \(\mathbb{R}^1\) must be in \(S\), since a scalar \(c = \frac{u_1}{v_1}\) exists
            such that \(u_1 = cv_1\) for all \(v_1 \neq 0\). Therefore, \(S\) = \(\mathbb{R}^1\).

        \item[(b)] Prove that a subspaces of \(\mathbb{R}^2\) is \(\mathbb{R}^2\), or the zero subspace,
            or consists of all scalar multiples of some fixed vector in \(\mathbb{R}^2\).
            (The last type of subspace is, intuitively, a straight line through the origin).
            
            Suppose \(v, u \in \mathbb{R}^2\) and \(v \neq 0\); if \(v = 0\) it can only span the trivial zero subspace.

            If \(v\) and \(u\) are scalar multiples of each other, they must span the subspace of
            set of all scalar multiples of some fixed vector in \(\mathbb{R}^2\).
            The space spanned by \(\{v\}\) is \(\{cv \mid c \in \mathbb{R}\}\).
            If \(u\) is scalar multiple of \(v\), then all scalar multiples of \(u\) are scalar multiples of \(v\) as well
            and therefore both \(\{u, v\}\) and \(\{v\}\) span the set of all scalar multiples of \(v\).

            Otherwise, if \(v\) and \(u\) are not scalar multiples of each other, they must span \(\mathbb{R}^2\).
            Let \(w\) be some linear combination of \(v = (v_1, v_2)\) and \(u = (u_1, u_2)\);
            \begin{gather*}
                \begin{bmatrix}
                    v_1 & u_1 \\
                    v_2 & u_2
                \end{bmatrix}
                X
                =
                w
            \end{gather*}

            Since \(v \neq 0\), suppose \(v_1 \neq 0\). Then
            \begin{gather*}
                \begin{bmatrix}
                    v_1 & u_1 \\
                    v_2 & u_2
                \end{bmatrix}
                \rightarrow
                \begin{bmatrix}
                    v_1 & u_1 \\
                    0 & \frac{v_1u_2 - u_1v_2}{v_1}
                \end{bmatrix}
            \end{gather*}

            If \(u\) is not a constant multiple of \(v\),
            then \((u_1, u_2) = (av_1, bv_2), a \neq b\) and \(v_1u_2 - u_1v_2 = au_1u_2 - bu_1u_2 \neq 0\).
            Then it's clear that the matrix is invertible and there must be a solution for all \(w\).
            If \(v_1 = 0\), then \(v_2 \neq 0\), and similar logic holds after swapping the two rows.
            This means vectors in \(\mathbb{R}^2\) can be written as a linear combination of \(v\) and \(u\),
            hence \(\{v, u\}\) must span \(\mathbb{R}^2\).

            (TODO: There's probably a more elegant way of doing this)

        \item[(c)] Can you describe the subspaces of \(\mathbb{R}^3\)?

            The only subspaces of \(\mathbb{R}^3\) are the zero subspace, \(\{cv \mid c \in \mathbb{R}\}\) for some \(v \in \mathbb{R}^3\),
            \(\{c_1v_1 + c_2v_2 \mid c_1, c_2 \in \mathbb{R}\}\) where \(v_1, v_2 \in \mathbb{R}^3\), and \(\mathbb{R}^3\) itself.

            (TODO)
    \end{enumerate}

\item[7.] Let \(W_1\) and \(W_2\) be subspaces of a vector space \(V\) such that the set-theoretic union of \(W_1\) and \(W_2\) is also a subspace.
    Prove that one of the spaces \(W_i\) is contained in the other.

    Suppose \(W_1 \cup W_2\) is a subspace, but \(W_2 \not\subset W_1\) and \(W_1 \not\subset W_2\).
    There is some vector \(v \in W_1\) such that \(v \not\in W_2\) and
    some vector \(u \in W_2\) such that \(u \not\in W_1\).
    Since vector spaces are closed under addition, \(v + u \in W_1 \cup W_2\), and either \(v + u \in W_1\) or \(v + u \in W_2\).
    This implies \(v + u + (-v) = u \in W_1\) or \(v + u + (-u) = v \in W_2\), which is a contradiction.
    Therefore, either \(W_1 \subset W_2\) or \(W_2 \subset W_1\).

\item[8.] Let \(V\) be the vector space of all functions from \(R\) into \(R\);
    let \(V_e\) be the subset of even functions, \(f(-x) = f(x)\);
    let \(V_o\) be the subset of odd functions, \(f(-x) = -f(x)\).

    \begin{enumerate}[listparindent=\parindent]
        \item[(a)] Prove that \(V_e\) and \(V_o\) are subspaces of \(V\).

            Suppose \(f, g \in V_e\), \(c \in \mathbb{R}\). Then
            \begin{gather*}
                (cf + g)(-x) \\
                = cf(-x) + g(-x) \\
                = cf(x) + g(x) \\
                = (cf(x) + g(x) \\
                = (cf + g)(x)
            \end{gather*}
            Since \((cf + g)(-x) = (cf + g)(x)\), \(V_e\) is closed under addition and multiplication
            and therefore is a subspace of \(V\).

            Likewise, suppose \(f, g \in V_o\), \(c \in \mathbb{R}\). Then
            \begin{gather*}
                (cf + g)(-x) \\
                = cf(-x) + g(-x) \\
                = -cf(x) + (-g(x)) \\
                = -(cf(x) + g(x)) \\
                = -(cf + g)(x)
            \end{gather*}
            Since \((cf + g)(-x) = -(cf + g)(x)\), \(V_o\) is closed under addition and multiplication
            and therefore is a subspace of \(V\).

        \item[(b)] Prove that \(V_e + V_o = V\).

            Any function \(f \in V\) can be written as a sum of functions \(g \in V_e\) and \(h \in V_o\);
            \begin{gather*}
                f(x) = g(x) + h(x) \\
                f(-x) = g(x) - h(x)
            \end{gather*}
            Add and subtract the two equations to get
            \begin{gather*}
                \frac{f(x) + f(-x)}{2} = g(x) \\
                \frac{f(x) - f(-x)}{2} = h(x)
            \end{gather*}
            which is defined for all \(f\).
            Therefore, any function in \(V\) is a sum of functions in \(V_e\) and \(V_o\); \(V = V_e + V_o\).

        \item[(c)] Prove that \(V_e \cap V_o = \{0\}\).

            If \(f \in V_e \cap V_o\), then \(f(-x) = f(x)\) and \(f(-x) = -f(x)\). Then
            \begin{gather*}
                f(x) = -f(x) \\
                2f(x) = 0 \\
                f(x) = 0
            \end{gather*}
    \end{enumerate}

\item[9.] Let \(W_1\) and \(W_2\) be subspaces of a vector space \(V\) such that \(W_1 + W_2 = V\)
    and \(W_1 \cup W_2 = \{0\}\). Prove that for each vector \(\alpha\) in \(V\) there are
    \textit{unique} vectors \(\alpha_1\) in \(W_1\) and \(\alpha_w\) in \(W_2\) such that \(\alpha = \alpha_1 + \alpha_2\).

    Suppose \(\alpha = \alpha_1 + \alpha_2\ = \beta_1 + \beta_2\) where \(\alpha_1, \beta_1 \in W_1\) and \(\alpha_2, \beta_2 \in W_2\).
    \begin{gather*}
        \alpha_1 - \beta_1 + \alpha_2 - \beta_2 = 0 \\
        \alpha_1 - \beta_1 = -(\alpha_2 - \beta_2) \\
        \alpha_1 - \beta_1 \in W_1 \\
        \beta_2 - \alpha_2 \in W_2 \\
        \implies \alpha_1 - \beta_1, \beta_2 - \alpha_2 \in W_1 \cap W_2
    \end{gather*}
    \(W_1 \cap W_2 = \{0\}\), so \(\alpha_1 - \beta_1 = \beta_2 - \alpha_2 = 0\)
    which must mean \(\alpha_1 = \beta_1\) and \(\alpha_2 = \beta_2\). 
    Therefore, \(\alpha_1\) and \(\alpha_2\) are unique.
\end{enumerate}

\subsection{Bases and Dimension}
\begin{enumerate}[listparindent=\parindent]
\end{enumerate}

\end{document}
