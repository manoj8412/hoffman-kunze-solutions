\documentclass{article}
\usepackage{array}
\usepackage{amssymb}
\usepackage{amsmath}
\usepackage{enumitem}
\usepackage{hyperref}
\hypersetup{
    bookmarksnumbered=true,
    pdfpagemode=UseOutlines,
}

\usepackage{array}

\setlist[enumerate]{font=\bfseries}

% The first argument is the number of columns before the divider,
% and the second argument is the number of columns before the divider,
\newenvironment{abmatrix}[2]
{
    \left[
        \begin{array} {@{} *{#1}{c} | *{#2}c @{}}
}
{
        \end{array}
    \right]
}


% The argument is the number of variables in the system.
\newcolumntype{B}{>{{}}c<{{}}}
\newenvironment{system}[1]
{
    \begin{cases}
        \setlength{\arraycolsep}{0pt}
        \begin{array} {r @{}*{#1}{B r}}
}
{
        \end{array}
    \end{cases}
}

\DeclareMathOperator{\rank}{rank}
\DeclareMathOperator{\nullity}{nullity}
\DeclareMathOperator{\Span}{span}
\DeclareMathOperator{\tr}{tr}


\begin{document}

\setcounter{section}{1}
\section{Vector Spaces}

\subsection{Vector Spaces}
\begin{enumerate}[listparindent=\parindent]
\item[1.] If \(F\) is a field, verify that \(F^n\) (as defined in Example 1) is a vector space over the field \(F\).

Let \(\alpha = (x_1, x_2, \dots, x_n)\), \(\beta = (y_1, y_2, \dots, y_n)\), and \(\gamma = (z_1, z_2, \dots, z_n)\).
First we prove the properties of vector addition, using the definition of addition over a field:
    \begin{enumerate}[listparindent=\parindent]
        \item[(a)] Addition is commutative.
        \begin{gather*}
            \alpha + \beta \\
            = (x_1 + y_1, x_2 + y_2, \dots, x_n + y_n) \\
            = (y_1 + x_1, y_2 + x_2, \dots, y_n + x_n) \\
            = \beta + \alpha
        \end{gather*}

        \item[(b)] Addition is associative.
        \begin{gather*}
            (\alpha + \beta) + \gamma \\
            = ((x_1 + y_1) + z_1, (x_2 + y_2) + z_2, \dots, (x_n + y_n) + z_n) \\
            = (x_1 + (y_1 + z_1), x_2 + (y_2 + z_2), \dots, x_n + (y_n + z_n)) \\
            = \alpha + (\beta + \gamma)
        \end{gather*}

        \item[(c)] There is a unique vector 0 such that \(\alpha + 0 = \alpha\) for all \(\alpha\) in V.

            Since \(F\) is a field, there exists a unique element 0 in \(F\) such that \(x + 0 = x\) for all \(x\) in \(F\),
            so there is a unique zero vector which is a \(n\)-tuple containing all 0s:
            \[(x_1 + 0, \dots, x_n + 0) = (x_1, \dots, x_n) + (0, \dots, 0) = \alpha + 0\]

        \item[(d)] For each vector \(\alpha\) in \(V\) there is a unique vector \(-\alpha\) in \(V\) such that \(\alpha + (-\alpha) = 0\).

            Since \(F\) is a field, there exists a unique element \(-x\) in \(F\) such that \(x + (-x) = x\) for all \(x\) in \(F\). Therefore,
            \begin{gather*}
                (x_1 + -(x_1), \dots, x_n + -(x_n)) \\
                = (x_1, \dots, x_n) + (-x_1, \dots, -x_n) \\
                = \alpha + -\alpha
            \end{gather*}
    \end{enumerate}

Then we prove the properties of scalar multiplication:
    \begin{enumerate}[listparindent=\parindent]
        \item[(a)] \(1\alpha = \alpha\) for every \(\alpha \in V\).
        \[1\alpha = (1x_1, \dots, 1x_n) = (x_1, \dots, x_n) = \alpha\]

        \item[(b)] \((c_1c_2)\alpha = c_1(c_2\alpha)\).
        \begin{gather*}
            (c_1c_2)\alpha = ((c_1c_2)x_1, \dots, (c_1c_2)x_n) \\
            = (c_1(c_2x_1), \dots, c_1(c_wx_n)) \\
            = c_1(c_2\alpha)
        \end{gather*}

        \item[(c)] \(c(\alpha + \beta) = c\alpha + c\beta\)
        \begin{gather*}
            c(\alpha + \beta) = c(x_1 + y_1, \dots, x_n + y_n) \\
            = (c(x_1 + y_1), \dots, c(x_n + y_n)) \\
            = (cx_1 + cy_1, \dots, cx_n + cy_n) \\
            = c\alpha + c\beta
        \end{gather*}

        \item[(d)] \((c_1 + c_2)\alpha = c_1\alpha + c_2\alpha\)
        \begin{gather*}
            (c_1 + c_2)\alpha = ((c_1 + c_2)x_1, \dots, (c_1 + c_2)x_n) \\
            = (c_1x_1 + c_2x_1, \dots, c_1x_n + c_2x_n) \\
            = c_1\alpha + c_2\alpha
        \end{gather*}
    \end{enumerate}

As it satisfies all the properties of a vector space, \(R^n\) is a vector space over \(F\).

\item[2.] If \(V\) is a vector space over the field \(F\), verify that
    \[(\alpha_1 + \alpha_2) + (\alpha_3 + \alpha_4) = [\alpha_2 + (\alpha_3 + \alpha_1)] + \alpha_4\]
    for all vectors \(\alpha_1, \alpha_2, \alpha_3, \alpha_4 \in V\).
    \begin{gather*}
        (\alpha_1 + \alpha_2) + (\alpha_3 + \alpha_4) \\
        = [(\alpha_2 + \alpha_1) + \alpha_3] + \alpha_4 \\
        = [\alpha_2 + (\alpha_1 + \alpha_3)] + \alpha_4 \\
        = [\alpha_2 + (\alpha_3 + \alpha_1)] + \alpha_4
    \end{gather*}

\item[3.] If \(C\) is the field of complex numbers, which vectors in \(C^3\) are linear combinations of
    \((1, 0, -1), (0, 1, 1)\) and \((1, 1, 1)\)?

Let \((y_1, y_2, y_3)\) be a vector formed by linear combinations of the three given vectors.
Then for some scalars \(x_1, x_2, x_3 \in C\),
\[
    \begin{bmatrix}
        1 & 0 & 1 \\
        0 & 1 & 1 \\
        -1 & 1 & 1 \\
    \end{bmatrix}
    \begin{bmatrix}
        x_1 \\ x_2 \\ x_3
    \end{bmatrix}
    =
    \begin{bmatrix}
        y_1 \\ y_2 \\ y_3
    \end{bmatrix}
\]
or in short, \(AX = Y\). Elementary row operations can show that \(A\) is row-equivalent to \(I\),
and therefore \(A\) is invertible. We can then solve the system; \(X = A^{-1}Y\).
This means any vector \(Y\) over \(C\) can be written as a linear combination of the three given vectors.

\item[4.] Let \(V\) be the set of all pairs \((x, y)\) of real numbers, and let \(F\) be the field of real numbers.
    Define

    \[(x, y) + (x_1, y_1) = (x + x_1, y + y_1)\]
    \[c(x, y) = (cx, y).\]
    Is \(V\), with these operations, a vector space over the field of real numbers?

\(V\) is not a vector space because there is no zero scalar \(0\) such that \(0(x, y) = 0\) for all \(y \in V\);
\(0(x, y) = (0x, y) \neq (0, 0)\) unless \(y = 0\).

\item[5.] On \(R^n\), define two operations
    \begin{gather*}
        \alpha \oplus \beta = \alpha - \beta \\
        c\cdot\alpha = -c\alpha
    \end{gather*}
    The operations on the right are the usual ones. Which of the axioms for a vector space are satisfied by \((R^n, \oplus, \cdot)\)?

\end{enumerate}
\end{document}
