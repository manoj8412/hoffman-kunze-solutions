\documentclass{article}
\usepackage{amssymb}
\usepackage{amsmath}
\usepackage{enumitem}

\begin{document}

\section{Linear Equations}

\setcounter{subsection}{1} % skip 1.1 because it has no exercises

\subsection{System of Linear Equations}
\paragraph{Exercises}

\begin{enumerate}[listparindent=\parindent]
\item[1.] Verify that the set of complex numbers of the form \(x + y\sqrt{2}\),
    where \(x\) and \(y\) are rational, is a subfield of \(\mathbb{C}\).

Let \(\mathbb F = \{x + y\sqrt{2} | x, y \in \mathbb Q\}\) and \(a, b, c, d \in \mathbb Q\). Then,

\[ (a + b\sqrt{2}) + (c + d\sqrt{2}) = (a + c) + (b + d)\sqrt{2} \]
\[ (a + b\sqrt{2})(c + d\sqrt{2}) = (ac + 2bd) + (bc + ad)\sqrt{2} \]

\(\mathbb Q\) is closed under addition and multiplication,
so the \(\mathbb F\) is also closed under addition and multiplication.

Addition and multiplication are also commutative and associative for \(\mathbb Q\),
so \(\mathbb F\) must also have commutative and associative addition and muliplication:

\[ (a + b\sqrt{2}) + (c + d\sqrt{2}) = (a + c) + (b + d)\sqrt{2} =  \]
\[ (c + a) + (d + b)\sqrt{2} = (c + d\sqrt{2}) + (a + b\sqrt{2}) \]

\[ (a + b\sqrt{2})(c + d\sqrt{2}) = (ac + 2bd) + (bc + ad)\sqrt{2} =  \]
\[ (ca + 2db\sqrt{2})(cb + da\sqrt{2}) = (c + d\sqrt{2})(a + b\sqrt{2}) \]

To prove there is a unique additive identity, \(0 = x + y\sqrt{2}\). Then,
\[ (a + b\sqrt{2}) + (x + y\sqrt{2}) = a + b\sqrt{2} = (a + x) + (b + y)\sqrt{2} \]
\[ x = y = 0 \]

Likewise, there is a unique multiplicative identity, \(1 = x + y\sqrt{2}\).
\[ (a + b\sqrt{2})(x + y\sqrt{2}) = a + b\sqrt{2} = (ax + 2by) + (bx + ay)\sqrt{2} \]
\[
    \begin{cases}
        \begin{alignedat}{2}
            ax + 2by &= a \\
            bx + ay &= b
        \end{alignedat}
    \end{cases}
 \]
\[ a(x - 1) + 2by = 0, b(x - 1) + ay = 0 \]
\[ x - 1 = \frac{-2by}{a} = \frac{-ay}{b} \]
\[ x = 1, y = 0 \]

Finally, there exists a unique additive inverse \(-x\) for all \(x \in \mathbb F\),
such that \(x + (-x) = 0\):
\[ (a + b\sqrt{2}) + (x + y\sqrt{2}) = (a + x) + (b + y)\sqrt{2} = 0 + 0\sqrt{2} \]
\[ a + x = 0, b + y = 0 \]
\[ x = -a, y = -b \]
\[ (a + b\sqrt{2}) + (-a + -b\sqrt{2}) = 0 + 0\sqrt{2} \]

and a unique multiplicative inverse \(x^{-1}\) for all \(x \in \mathbb F\) and \(x \neq 0\),
such that \(x(x^{-1}) = 1\):
\[ (a + b\sqrt{2})(x + y\sqrt{2}) = (ax + 2by) + (bx + ay)\sqrt{2} = 1 + 0\sqrt{2}  \]

Then,
\[
    \begin{cases}
        \begin{alignedat}{2}
            ax + 2by &= 1 \\
            bx + ay &= 0
        \end{alignedat}
    \end{cases}
\]
which can be solved to get \(x = \frac{a}{a^2 - 2b^2}, y = \frac{-b}{a^2 - 2b^2}\).

Therefore, \(\mathbb F\) follows all the axioms of a field,
and therefore is a field and a subfield of \(\mathbb C\).

\item[2.] Let \(\mathbb F\) be a field of complex numbers. Are the following two systems of linear equations equivalent?
    If so, express each equation in each system as a linear combination of the equations in another system.
    \[
        \begin{cases}
            \begin{alignedat}{2}
                x_1 - x_2 &= 0 \\
                2x_1 + x_2 &= 0
            \end{alignedat}
        \end{cases}
        \quad
        \begin{cases}
            \begin{alignedat}{2}
                3x_1 + x_2 &= 0\\
                x_1 + x_2 &= 0
            \end{alignedat}
        \end{cases}
     \]

The two systems of equations are equivalent.
Each equation in each system can be written as a linear combination of the equations in another system:
\[
    \begin{cases}
        \begin{alignedat}{2}
                         (3x_1+x_2) -            2 (x_1+x_2) &=  x_1-x_2 \\
            \tfrac{1}{2} (3x_1+x_2) + \tfrac{1}{2} (x_1+x_2) &= 2x_1+x_2 \\
            \tfrac{4}{3} (2x_1+x_2) + \tfrac{1}{3} (x_1-x_2) &= 3x_1+x_2 \\
            \tfrac{2}{3} (2x_1+x_2) - \tfrac{1}{3} (x_1-x_2) &=  x_1+x_2
        \end{alignedat}
    \end{cases}
 \]

\item[3.] Test the following systems as in Exercise 2.
    \[
        \begin{cases}
            \begin{alignedat}{4}
                -x_1 &+ x_2 &+ 4x_3 &= 0 \\
                x_1 &+ 3x_2 &+ 8x_3 &= 0 \\
                \tfrac{1}{2}x_1 &+ x_2 &+ \tfrac{5}{2}x_3 &= 0
            \end{alignedat}
        \end{cases}
        \quad
        \begin{cases}
            \begin{alignedat}{3}
                x_1 &- x_3 &= 0 \\
                x_2 &+ 3x_3 &= 0
            \end{alignedat}
        \end{cases}
     \]

\[ -(x_1 - x_3) + (x_2 + 3x_3) = -x_1 + x_2 + 4x_3 \]
\[ (x_1 - x_3) + 3(x_2 + 3x_3) = x_1 + 3x_2 + 8x_3 \]
\[ \frac{1}{2}(x_1 - x_3) + (x_2 + 3x_3) = \frac{1}{2}x_1 + x_2 + \frac{5}{2}x_3 \]

\[ 0(-x_1 + x_2 + 4x_3) + -2(x_1 + 3x_2 + 8x_3) + 6(\frac{1}{2}x_1 + x_2 + \frac{5}{2}x_3) = x_1 - x_3 \]
\[ 0(-x_1 + x_2 + 4x_3) + (x_1 + 3x_2 + 8x_3) + -2(\frac{1}{2}x_1 + x_2 + \frac{5}{2}x_3) = x_2 + 3x_3 \]

\item[6.] Prove that if two homogeneous systems of linear equations in two unknowns have the same solutions,
    then they are equivalent.

Let
\[
    \begin{cases}
        \begin{alignedat}{2}
            a_{11}x + a_{12}x &= 0 \\
            a_{21}x + a_{22}x &= 0
        \end{alignedat}
    \end{cases}
    \quad
    \begin{cases}
        \begin{alignedat}{2}
            b_{11}x + b_{12}x &= 0 \\
            b_{21}x + b_{22}x &= 0
        \end{alignedat}
    \end{cases}
 \]
where \((x, y)\) is a solution to the systems.

The systems are homogeneous if there exist \(c_1\) and \(c_2\) such that
\[ c_1(a_{11}x + a_{12}y) + c_2(a_{21}x + a_{22}y) = b_{11}x + b_{12}y \]

Then it must be true that
\[ a_{11}c_1 + a_{21}c_2 = b_{11} \]
\[ a_{12}c_1 + a_{22}c_2 = b_{12} \]

Solve for \(c_1\) and \(c_2\):

\[ c_2 = \frac{b_{11} - a_{11}c_1} {a_{21}} \]
\[ a_{12}c_1 + \frac{a_{22}b_{11} - a_{22}a_{21}c_1} {a_{21}} = b_{12} \]
\[ \frac{a_{22}b_{11} - a_{22}a_{21}c_1 + a_{12}a_{21}c_1} {a_{21}} = b_{12} \]
\[ a_{22}b_{11} - a_{22}a_{21}c_1 + a_{12}a_{21}c_1 = a_{21}b_{12} \]
\[ c_1 = \frac{a_{12}b_{21} - a_{22}b_{21}} {a_{11}a_{22} - a_{12}a_{21}} \]

Similarly,
\[ c_2 = \frac{a_{11}b_{12} - a_{12}b_{11}} {a_{11}a_{22} - a_{12}a_{21}} \]

When \(c_1\) and \(c_2\) are defined, one system can be written
as a linear combination of the another and the two systems are equivalent.
If \(c_1\) and \(c_2\) are undefined, then
\( a_{11}a_{22} - a_{12}a_{21} = 0 \).
\[ \frac{a_{11}}{a_{12}} = \frac{a_{21}}{a_{22}} \]

Then the two equations in the first system are multiples of each other, and the solution set is a line.
Since the second system has the same line as its solution set,
its equations are constant multiples of equations of the first system.
Therefore, the two systems are equivalent.

\item[7.] Prove that each subfield of the field of complex numbers contains every rational number.

Let \(\mathbb F\) be a subfield of \(\mathbb C\).
There exists a unique non-zero element \(1\) in \(\mathbb F\).
\(\mathbb F\) is closed under addition, so if \(x\) is an element of \(\mathbb F\) then \(x + 1\) must be as well.
Therefore, \(\mathbb Z^+ \in \mathbb F\).

For each nonzero element \(x\) of \(\mathbb F\), there exists a unique element \(-x\) such that \(x + (-x)\ = 0\).
Therefore, \(-1 \in \mathbb F\).
\(\mathbb F\) is closed under multiplication, so for any positive integer \(x\), \(-x\) is also an element of \(\mathbb F\),
so \(\mathbb Z^- \in \mathbb F\).
There also exists a unique element \(0\) in \(\mathbb F\), such that \(x0 = 0\).
Therefore, \(\mathbb Z \subseteq \mathbb F\).

Then, let \(n\) and \(m\) be two integers, where \(n \neq 0\).
\(\mathbb F\) must also contain the of multiplicative inverse of \(n\), or \(\frac{1}{n}\), such that \(n\frac{1}{n} = 1\).
A field is closed under multiplication, so \(\frac{m}{n} \in \mathbb F\),
for any integers \(n\) and \(m\), where \(n \neq 0\).
Therefore, \(\mathbb Q \subseteq \mathbb F\).

\end{enumerate}

\subsection{Matrices and Elementary Row Operations}
\paragraph{Exercises}

\begin{enumerate}[listparindent=\parindent]

\item[1.] Find all solutions to the system of equations
    \[
        \begin{cases}
            \begin{alignedat}{2}
                (1 - i)x_1 - ix_2 &= 0 \\
                2x_1 + (1 - i)x_2 &= 0
            \end{alignedat}
        \end{cases}
    \]

\[
    \begin{bmatrix}
        1 - i & -i \\
        2 & i - 1
    \end{bmatrix}
    \xrightarrow{(1)}
    \begin{bmatrix}
        1 & \frac{i - 1}{2} \\
        2 & i - 1
    \end{bmatrix}
    \xrightarrow{(2)}
    \begin{bmatrix}
        1 & \frac{i - 1}{2} \\
        0 & 0
    \end{bmatrix}
\]

\[ x_1 + \frac{1 - i}{2}x_2 = 0 \]

\[
    \boxed{
        (\frac{i - 1}{2}, x) \forall x \in \mathbb{C}
    }
\]

\item[2.] If
    \[
        A = \begin{bmatrix}
            3 & -1 & 2 \\
            2 & 1 & 1 \\
            1 & -3 & 0
        \end{bmatrix}
    \]
    find all solutions of \(AX = 0\) by row-reducing \(A\).

\[
    \begin{bmatrix}
        3 & -1 & 2 \\
        2 & 1 & 1 \\
        1 & -3 & 0
    \end{bmatrix}
    \xrightarrow{(2)}
    \begin{bmatrix}
        0 & 8 & 2 \\
        0 & 7 & 1 \\
        1 & -3 & 0
    \end{bmatrix}
    \xrightarrow{(1)}
    \begin{bmatrix}
        0 & 8 & 2 \\
        0 & 1 & \frac{1}{7} \\
        1 & -3 & 0
    \end{bmatrix}
    \xrightarrow{(2)}
    \begin{bmatrix}
        0 & 0 & \frac{6}{7} \\
        0 & 1 & \frac{1}{7} \\
        1 & 0 & \frac{3}{7}
    \end{bmatrix}
\]
\[
    \xrightarrow{(1)}
    \begin{bmatrix}
        0 & 0 & 1 \\
        0 & 1 & \frac{1}{7} \\
        1 & 0 & \frac{3}{7}
    \end{bmatrix}
    \xrightarrow{(2)}
    \begin{bmatrix}
        0 & 0 & 1 \\
        0 & 1 & 0 \\
        1 & 0 & 0
    \end{bmatrix}
\]
\[
\boxed{ (0, 0, 0) }
\]

\item[3.] If
    \[
        A = \begin{bmatrix}
            6 & -4 & 0 \\
            4 & -2 & 0 \\
            -1 & 0 & 3
        \end{bmatrix}
    \]
    find all solutions of \(AX = 2X\) and \(AX = 3X\).
    (The symbol \(cX\) denotes the matrix each entry of which is \(c\) times the corresponding entry of \(X\).)

\begin{enumerate}
\item[(a)] \(AX = 2X\)

    This is equivalent to the system
    \[
        \begin{cases}
            \begin{alignedat}{2}
                6x - 4y &= 2x \\
                4x - 2y &= 2y \\
                -x + 3z &= 2z
            \end{alignedat}
        \end{cases}
    \]
    which can be written as a homogeneous system
    \[
        \begin{cases}
            \begin{alignedat}{2}
                4x - 4y &= 0 \\
                4x - 4y &= 0 \\
                -x + z &= 0
            \end{alignedat}
        \end{cases}
    \]

    The system can be represented as
    \[
        \begin{bmatrix}
            4 & -4 & 0 \\
            4 & -4 & 0 \\
            -1 & 0 & 1
        \end{bmatrix}
        \xrightarrow{(1)}
        \begin{bmatrix}
            1 & -1 & 0 \\
            1 & -1 & 0 \\
            -1 & 0 & 1
        \end{bmatrix}
        \xrightarrow{(2)}
        \begin{bmatrix}
            1 & -1 & 0 \\
            0 & 0 & 0 \\
            0 & -1 & 1
        \end{bmatrix}
    \]
    \[
        \xrightarrow{(1)}
        \begin{bmatrix}
            1 & -1 & 0 \\
            0 & 0 & 0 \\
            0 & 1 & -1
        \end{bmatrix}
        \xrightarrow{(2)}
        \begin{bmatrix}
            1 & 0 & -1 \\
            0 & 0 & 0 \\
            0 & 1 & -1
        \end{bmatrix}
    \]

    \[ \boxed {
        (x, x, x) \forall x \in \mathbb C.
    } \]

\item[(b)] \(AX = 3X\)
    Similarly to the previous question, this system is equivalent to
    \[
        \begin{bmatrix}
            3 & -4 & 0 \\
            4 & -5 & 0 \\
            -1 & 0 & 0
        \end{bmatrix}
    \]

    The matrix can be reduced to
    \[
        \begin{bmatrix}
            1 & 0 & 0 \\
            0 & 1 & 0 \\
            0 & 0 & 0
        \end{bmatrix}
    \]
    \[ \boxed {
        (0, 0, x) \forall x \in \mathbb C.
    } \]
\end{enumerate}

\item[4.] Find a row-reduced matrix which is row-equivalent to
    \[
        A = \begin{bmatrix}
            i & -(1 + i) & 0 \\
            1 & -2 & 1 \\
            1 & 2i & -1
        \end{bmatrix}
    \]

\[
    \begin{bmatrix}
        i & -1 - i & 0 \\
        1 & -2 & 1 \\
        1 & 2i & -1
    \end{bmatrix}
    \xrightarrow{(2)}
    \begin{bmatrix}
        0 & 1 - i & i \\
        0 & -2-2i & 2 \\
        1 & 2i & -1
    \end{bmatrix}
    \xrightarrow{(1)}
    \begin{bmatrix}
        0 & 1 - i & i \\
        0 & 1 & \frac{i - 1}{2} \\
        1 & 2i & -1
    \end{bmatrix}
\]
\[
    \xrightarrow{(2)}
    \boxed{
        \begin{bmatrix}
            0 & 0 & 0 \\
            0 & 1 & \frac{i - 1}{2} \\
            1 & 0 & i
        \end{bmatrix}
    }
\]

\item[5.] Prove that the following two matrices are \textit{not} row-equivalent:
    \[
        \begin{bmatrix}
            2 & 0 & 0 \\
            a & -1 & 0 \\
            b & c & 3
        \end{bmatrix}
        \quad
        \begin{bmatrix}
            1 & 1 & 2 \\
            -2 & 0 & -1 \\
            1 & 3 & 5
        \end{bmatrix}
    \]

The row-reduced matrix which is row-equivalent to the first matrix is
\[
    \begin{bmatrix}
        2 & 0 & 0 \\
        a & -1 & 0 \\
        b & c & 3
    \end{bmatrix}
    \xrightarrow{(1)}
    \begin{bmatrix}
        1 & 0 & 0 \\
        0 & -1 - 2a & 0 \\
        0 & c - 2b & 3
    \end{bmatrix}
    \xrightarrow{(2)}
    \begin{bmatrix}
        1 & 0 & 0 \\
        0 & -1 - 2a & 0 \\
        0 & c - 2b & 3
    \end{bmatrix}
\]
\[
    \xrightarrow{(1)}
    \begin{bmatrix}
        1 & 0 & 0 \\
        0 & 1 & 0 \\
        0 & c - 2b & 3
    \end{bmatrix}
    \xrightarrow{(2)}
    \begin{bmatrix}
        1 & 0 & 0 \\
        0 & 1 & 0 \\
        0 & 0 & 3 - (c - 2b)
    \end{bmatrix}
    \xrightarrow{(1)}
    \boxed{
        \begin{bmatrix}
            1 & 0 & 0 \\
            0 & 1 & 0 \\
            0 & 0 & 1
        \end{bmatrix}
    }
\]
which has \( (0, 0, 0) \) as its solution.

The second matrix is equivalent to
\[
    \begin{bmatrix}
        1 & 1 & 2 \\
        -2 & 0 & -1 \\
        1 & 3 & 5
    \end{bmatrix}
    \xrightarrow{(2)}
    \begin{bmatrix}
        1 & 1 & 2 \\
        0 & 2 & 3 \\
        0 & 2 & 3
    \end{bmatrix}
    \xrightarrow{(1)}
    \begin{bmatrix}
        1 & 1 & 2 \\
        0 & 1 & \frac{3}{2} \\
        0 & 2 & 3
    \end{bmatrix}
    \xrightarrow{(2)}
    \boxed{
        \begin{bmatrix}
            1 & 0 & \frac{1}{2} \\
            0 & 1 & \frac{3}{2} \\
            0 & 0 & 0
        \end{bmatrix}
    }
\]
which has \( (\frac{-1}{2}x, \frac{-3}{2}x, x) \forall x \in \mathbb C \) as its solution set.

By Theorem 3, if the two matrices are row-equivalent,
then the two homogeneous systems represented by the matrices have exactly the same solutions.
However, the first has only one solution and the second has an infinite amount of solutions.
The homogeneous systems do not have the same solutions and the two matrices cannot be row-equivalent.

\item[6.] Let
    \[
        A = \begin{bmatrix}
            a & b \\ c & d
        \end{bmatrix}
    \]
    be a \(2 \times 2\) matrix with complex entries. Suppose that \(A\) is row-reduced and also that \(a + b + c + d = 0\).
    Prove that there are exactly three such matrices.

Assume \(A\) is non-zero matrix.
Suppose the first column of \(A\) is a zero column.
Since the matrix is non-zero, there must be non-zero entry in the second column. So either \(b\) or \(d\) must be non-zero.
This non-zero entry is also the first non-zero entry, meaning it must be 1. Let exactly one of them be 1.
Then the other must be zero because other entries of a column containing a leading non-zero entry must be zero.
Then \(a + b + c + d = 1 \neq 0\). 

Therefore, \(A\) must be in the form \( (\begin{smallmatrix} 1 & b \\ 0 & d \end{smallmatrix}) \)
or \( (\begin{smallmatrix} 0 & b \\ 1 & d \end{smallmatrix}) \).

Consider the case where \(A = (\begin{smallmatrix} 1 & b \\ 0 & d\end{smallmatrix}) \).
If \(d\) is a leading one, \(b\) must be zero because it is in the same column as \(d\).
But then \(a + b + c + d = 2 \neq 0\).
Thus \(d\) cannot be a leading non-zero entry, and must be zero.
Therefore \(a + b + c + d = 1 + b + 0 + 0 = -1\), and \(b = -1\), meaning
\[
    A = \begin{bmatrix}
        1 & -1 \\ 0 & 0
    \end{bmatrix}
\]
is only possible case. The same logic follows for the case where \(A = (\begin{smallmatrix} 0 & b \\ 1 & d\end{smallmatrix}) \),
leading to 
\[
    A = \begin{bmatrix}
        0 & 0 \\ 1 & -1
    \end{bmatrix}
\]
as the only case. Finally, there is the trivial zero matrix as our third possiblity.

Therefore, the only three possible matrices are
\[
    \begin{bmatrix}
        0 & 0 \\ 0 & 0
    \end{bmatrix},
    \begin{bmatrix}
        1 & -1 \\ 0 & 0
    \end{bmatrix},
    \begin{bmatrix}
        0 & 0 \\ 1 & -1
    \end{bmatrix}
\]

\item[7.] Prove that the interchange of two rows of a matrix can be accomplished by a finite sequence of
    elementary row operations of the other two types.

Let \(R_i\) and \(R_j\) be two rows of a matrix. Then
\[
    \begin{bmatrix}
        R_i \\ R_j
    \end{bmatrix}
    \xrightarrow{(2)}
    \begin{bmatrix}
        R_i + R_j \\ R_j
    \end{bmatrix}
    \xrightarrow{(2)}
    \begin{bmatrix}
        R_i + R_j \\ -R_i
    \end{bmatrix}
    \xrightarrow{(2)}
    \begin{bmatrix}
        R_j \\ -R_i
    \end{bmatrix}
    \xrightarrow{(1)}
    \begin{bmatrix}
        R_j \\ R_i
    \end{bmatrix}
\]

\item[8.] Consider the system of equations \(AX = 0\) where
    \[
        A = \begin{bmatrix}
            a & b \\ c & d
        \end{bmatrix}
    \]
    is a \(2 \times 2\) matrix over the field \(F\). Prove the following.

\begin{enumerate}
\item[(a)] If every entry of \(A\) is 0, then every pair \(x_1, x_2\) is a solution of \(AX = 0\).
This represents the following system of equations:
\[
    \begin{cases}
        \begin{alignedat}{2}
            0x_1 + 0x_2 &= 0 \\
            0x_1 + 0x_2 &= 0
        \end{alignedat}
    \end{cases}
\]
\(0x \forall x \in F\) and \(0 + 0 = 0\), so every pair \(x_1, x_2\) is a solution.

\item[(b)] If \(ad - bc \neq 0\), the system \(AX = 0\) has only the trivial solution \(x_1 = x_2 = 0\).

\textbf{Case 1:} \(a \neq 0\)

\[
    \begin{bmatrix}
        a & b \\
        c & d
    \end{bmatrix}
    \xrightarrow{(1)}
    \begin{bmatrix}
        1 & \frac{b}{a} \\
        c & d
    \end{bmatrix}
    \xrightarrow{(2)}
    \begin{bmatrix}
        1 & \frac{b}{a} \\
        0 & d - \frac{bc}{a}
    \end{bmatrix}
    =
    \begin{bmatrix}
        1 & \frac{a}{b} \\
        0 & \frac{ad - bc}{a}
    \end{bmatrix}
\]

\(ad - bc \neq 0\), so \( \frac{a}{ad - bc} \) is defined. Multiplying second row with it leads to the identity matrix:
\[
    \begin{bmatrix}
        1 & \frac{a}{b} \\
        0 & \frac{ad - bc}{a}
    \end{bmatrix}
    \xrightarrow{(1)}
    \begin{bmatrix}
        1 & \frac{a}{b} \\
        0 & 1
    \end{bmatrix}
    \xrightarrow{(2)}
    \begin{bmatrix}
        1 & 0 \\
        0 & 1
    \end{bmatrix}
\]
Then \(A = I_2\) and \(AX = 0\) only has the trivial solution \(x_1 = x_2 = 0\).

\textbf{Case 2:} \(a = 0\)

Note that \(b \neq 0\) and \(c \neq 0\) because otherwise \(ad - bc = 0\).
\[
    \begin{bmatrix}
        0 & b \\
        c & d
    \end{bmatrix}
    \xrightarrow{(1)}
    \begin{bmatrix}
        0 & 1 \\
        1 & \frac{d}{c}
    \end{bmatrix}
    \xrightarrow{(2)}
    \begin{bmatrix}
        0 & 1 \\
        1 & 0
    \end{bmatrix}
\]
Again, \(A = I_2\) and \(AX = 0\) only has the trivial solution \(x_1 = x_2 = 0\).

\item[(c)] If \(ad - bc = 0\), and some entry of \(A\) is different from 0, then there is a solution
    \( (x_1^0, x_2^0) \) such that \( (x_1, x_2) \) is a solution if and only if there is a scalar y such that
    \( x_1 = yx_1^0, x_2 = yx_2^0 \).

Assume that \(a \neq 0\), since \(A\) has at least one non-zero entry. Then \(d = \frac{bc}{a}\).
\[
    \begin{bmatrix}
        a & b \\
        c & \frac{bc}{a}
    \end{bmatrix}
    \xrightarrow{(2)}
    \begin{bmatrix}
        a & b \\
        0 & \frac{bc}{a} - \frac{bc}{a}
    \end{bmatrix}
    =
    \begin{bmatrix}
        a & b \\
        0 & 0
    \end{bmatrix}
\]
Then \(AX = Y\) has the solution set of \( (x, \frac{-a}{b}x) \forall x \in F \),
and any solution must be a scalar multiple of \( (1, \frac{-a}{b}) \).
If any other entry is non-zero, a similar argument can be given.
\end{enumerate}
\end{enumerate}

\subsection{Row-Reduced Echelon Matrices}
\paragraph{Exercises}

\begin{enumerate}[listparindent=\parindent]
\item[1.] Find all solutions to the following system of equations by row-reducing the coefficient matrix:
    \[
        \begin{cases}
            \begin{alignedat}{6}
                \tfrac{1}{3}x_1 &+& 2x_2 &-& 6x_3 &= 0 \\
                -4x_1 && &+& 5x_3 &= 0 \\
                -3x_1 &+& 6x_2 &-& 13x_3 &= 0 \\
                -\tfrac{7}{3}x_1 &+& 2x_2 &-& \tfrac{8}{3}x_3 &= 0 \\
            \end{alignedat}
        \end{cases}
    \]
\[
    \begin{bmatrix}
        \frac{1}{3} & 2 & -6 \\
        -4 & 0 & 5 \\
        -3 & 6 & -13 \\
        -\frac{7}{3} & 2 & -\frac{8}{3}
    \end{bmatrix}
    \rightarrow
    \begin{bmatrix}
        1 & 6 & -18 \\
        -4 & 0 & 5 \\
        -3 & 6 & -13 \\
        -7 & 6 & -8
    \end{bmatrix}
    \rightarrow
    \begin{bmatrix}
        1 & 6 & -18 \\
        0 & 24 & -67 \\
        0 & 24 & -67 \\
        0 & 48 & -134
    \end{bmatrix}
\]
\[
    \rightarrow
    \begin{bmatrix}
        1 & 6 & -18 \\
        0 & 24 & -67 \\
        0 & 0 & 0 \\
        0 & 0 & 0
    \end{bmatrix}
    \rightarrow
    \begin{bmatrix}
        1 & 6 & -18 \\
        0 & 1 & -\frac{67}{24} \\
        0 & 0 & 0 \\
        0 & 0 & 0
    \end{bmatrix}
    \rightarrow
    \begin{bmatrix}
        1 & 0 & -\frac{5}{4} \\
        0 & 1 & -\frac{67}{24} \\
        0 & 0 & 0 \\
        0 & 0 & 0
    \end{bmatrix}
\]
\[
    \boxed {
        (\frac{5}{4}x, \frac{67}{24}x, x) \forall x \in F
    }
\]

\item[2.] Find a row-reduced echelon matrix which is row-equivalent to
    \[
        \begin{bmatrix}
            1 & -i \\
            2 & 2 \\
            i & 1 + i 
        \end{bmatrix}
    \]

\end{enumerate}

\end{document}
