\documentclass{article}
\usepackage{amssymb}
\usepackage{amsmath}
\usepackage{enumitem}
\usepackage{systeme}

\begin{document}

\section{Linear Equations}

\setcounter{subsection}{1}
\subsection{System of Linear Equations}

\paragraph{Exercises}
\begin{enumerate}[listparindent=\parindent]
\item[1.] Verify that the set of complex numbers of the form \(x + y\sqrt{2}\), where \(x\) and \(y\) are rational, is a subfield of \(\mathbb{C}\).

    Let \(\mathbb F = \{x + y\sqrt{2} | x, y \in \mathbb Q\}\) and \(a, b, c, d \in \mathbb Q\). Then,

    \[(a + b\sqrt{2}) + (c + d\sqrt{2}) = (a + c) + (b + d)\sqrt{2}\]
    \[(a + b\sqrt{2})(c + d\sqrt{2}) = (ac + 2bd) + (bc + ad)\sqrt{2}\]

    \(\mathbb Q\) is closed under addition and multiplication,
    so the \(\mathbb F\) is also closed under addition and multiplication.

    Addition and multiplication are also commutative and associative for \(\mathbb Q\),
    so \(\mathbb F\) must also have commutative and associative addition and muliplication:

    \[(a + b\sqrt{2}) + (c + d\sqrt{2}) = (a + c) + (b + d)\sqrt{2} = \]
    \[(c + a) + (d + b)\sqrt{2} = (c + d\sqrt{2}) + (a + b\sqrt{2})\]

    \[(a + b\sqrt{2})(c + d\sqrt{2}) = (ac + 2bd) + (bc + ad)\sqrt{2} = \]
    \[(ca + 2db\sqrt{2})(cb + da\sqrt{2}) = (c + d\sqrt{2})(a + b\sqrt{2})\]

    To prove there is a unique additive identity, \(0 = x + y\sqrt{2}\). Then,
    \[(a + b\sqrt{2}) + (x + y\sqrt{2}) = a + b\sqrt{2} = (a + x) + (b + y)\sqrt{2}\]
    \[x = y = 0\]

    Likewise, there is a unique multiplicative identity, \(1 = x + y\sqrt{2}\).
    \[(a + b\sqrt{2})(x + y\sqrt{2}) = a + b\sqrt{2} = (ax + 2by) + (bx + ay)\sqrt{2}\]
    \[
        \systeme{
            ax + 2by = a,
            bx + ay = b
        }
    \]
    \[a(x - 1) + 2by = 0, b(x - 1) + ay = 0\]
    \[x - 1 = \frac{-2by}{a} = \frac{-ay}{b}\]
    \[x = 1, y = 0\]

    Finally, there exists a unique additive inverse \(-x\) for all \(x \in \mathbb F\),
    such that \(x + (-x) = 0\):
    \[(a + b\sqrt{2}) + (x + y\sqrt{2}) = (a + x) + (b + y)\sqrt{2} = 0 + 0\sqrt{2}\]
    \[a + x = 0, b + y = 0\]
    \[x = -a, y = -b\]
    \[(a + b\sqrt{2}) + (-a + -b\sqrt{2}) = 0 + 0\sqrt{2}\]

    and a unique multiplicative inverse \(x^{-1}\) for all \(x \in \mathbb F\) and \(x \neq 0\),
    such that \(x(x^{-1}) = 1\):
    \[(a + b\sqrt{2})(x + y\sqrt{2}) = (ax + 2by) + (bx + ay)\sqrt{2} = 1 + 0\sqrt{2} \]

    Then,
    \[
        \systeme{
            ax + 2by = 1,
            bx + ay = 0
        }
    \]
    which can be solved to get \(x = \frac{a}{a^2 - 2b^2}, y = \frac{-b}{a^2 - 2b^2}\).

    Therefore, \(\mathbb F\) follows all the axioms of a field,
    and therefore is a field and a subfield of \(\mathbb C\).

\item[2.] Let \(\mathbb F\) be a field of complex numbers. Are the following two systems of linear equations equivalent?
    If so, express each equation in each system as a linear combination of the equations in another system.
    \[
        \systeme{
            x_1 - x_2 = 0,
            2x_1 + x_2 = 0
        }
        \quad
        \systeme{
            3x_1 + x_2 = 0,
            x_1 + x_2 = 0
        }
    \]

    The two systems of equations are equivalent.
    Each equation in each system can be written as a linear combination of the equations in another system:
    \[
        \begin{alignedat}{2}
                        (3x_1+x_2) -           2 (x_1+x_2) &=  x_1-x_2 \\
            \frac{1}{2} (3x_1+x_2) + \frac{1}{2} (x_1+x_2) &= 2x_1+x_2 \\
            \frac{4}{3} (2x_1+x_2) + \frac{1}{3} (x_1-x_2) &= 3x_1+x_2 \\
            \frac{2}{3} (2x_1+x_2) - \frac{1}{3} (x_1-x_2) &=  x_1+x_2
        \end{alignedat}
    \]

\item[3.] Test the following systems as in Exercise 2.
    \[
        \systeme{
            -x_1 + x_2 + 4x_3 = 0,
            x_1 + 3x_2 + 8x_3 = 0,
            \frac{1}{2}x_1 + x_2 + \frac{5}{2}x_3 = 0
        }
        \quad
        \systeme{
            x_1 - x_3 = 0,
            x_2 + 3x_3 = 0
        }
    \]

\item[6.] Prove that if two homogeneous systems of linear equations in two unknowns have the same solutions,
    then they are equivalent.

    Let
    \[
        \systeme{
            xa_{11} + ya_{12} = 0,
            xa_{21} + ya_{22} = 0
        }
        \quad
        \systeme{
            xb_{11} + yb_{12} = 0,
            xb_{21} + yb_{22} = 0
        }
    \]
    where \((x, y)\) is a solution to the systems.

    The systems are homogeneous if there exist \(c_1\) and \(c_2\) such that
    \[c_1(xa_{11} + ya_{12}) + c_2(xa_{21} + ya_{22}) = xb_{11} + yb_{12}\]

    Then it must be true that
    \[a_{11}c_1 + a_{21}c_2 = b_{11}\]
    \[a_{12}c_1 + a_{22}c_2 = b_{12}\]

    Solve for \(c_1\) and \(c_2\):

    \[c_2 = \frac{b_{11} - a_{11}c_1} {a_{21}}\]
    \[a_{12}c_1 + \frac{a_{22}b_{11} - a_{22}a_{21}c_1} {a_{21}} = b_{12}\]
    \[\frac{a_{22}b_{11} - a_{22}a_{21}c_1 + a_{12}a_{21}c_1} {a_{21}} = b_{12}\]
    \[a_{22}b_{11} - a_{22}a_{21}c_1 + a_{12}a_{21}c_1 = a_{21}b_{12}\]
    \[c_1 = \frac{a_{12}b_{21} - a_{22}b_{21}} {a_{11}a_{22} - a_{12}a_{21}}\]

    Similarly,
    \[c_2 = \frac{a_{11}b_{12} - a_{12}b_{11}} {a_{11}a_{22} - a_{12}a_{21}}\]

    When \(c_1\) and \(c_2\) are defined, one system can be written
    as a linear combination of the another and the two systems are equivalent.
    If \(c_1\) and \(c_2\) are undefined, then
    \(a_{11}a_{22} - a_{12}a_{21} = 0\).
    \[\frac{a_{11}}{a_{12}} = \frac{a_{21}}{a_{22}}\]

    Then the two equations in the first system are multiples of each other, and the solution set is a line.
    Since the second system has the same line as its solution set,
    its equations are constant multiples of equations of the first system.
    Therefore, the two systems are equivalent.

\item[7.] Prove that each subfield of the field of complex numbers every rational number.

Let \(\mathbb F\) be a subfield of \(\mathbb C\).
There exists a unique non-zero element \(1\) in \(\mathbb F\).
\(\mathbb F\) is closed under addition, so if \(x\) is an element of \(\mathbb F\) then \(x + 1\) must be as well.
Therefore, \(\mathbb Z^+ \in \mathbb F\).

For each nonzero element \(x\) of \(\mathbb F\), there exists a unique element \(-x\) such that \(x + (-x)\ = 0\).
Therefore, \(-1 \in \mathbb F\).
\(\mathbb F\) is closed under multiplication, so for any positive integer \(x\), \(-x\) is also an element of \(\mathbb F\), 
so \(\mathbb Z^- \in \mathbb F\).
There also exists a unique element \(0\) in \(\mathbb F\), such that \(x0 = 0\).
Therefore, \(\mathbb Z \subseteq \mathbb F\).

Then, let \(n\) and \(m\) be two integers, where \(n \neq 0\).
\(\mathbb F\) must also contain the of multiplicative inverse of \(n\), or \(\frac{1}{n}\), such that \(n\frac{1}{n} = 1\).
A field is closed under multiplication, so \(\frac{m}{n} \in \mathbb F\),
for any integers \(n\) and \(m\), where \(n \neq 0\).
Therefore, \(\mathbb Q \subseteq \mathbb F\).

\item[8.]
    Prove that each field of characteristic zero contains a copy of the rational number field.

todo

\end{enumerate}

\end{document}
