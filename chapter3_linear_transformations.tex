\documentclass{article}
\usepackage{array}
\usepackage{amssymb}
\usepackage{amsmath}
\usepackage{enumitem}
\usepackage{hyperref}
\hypersetup{
    bookmarksnumbered=true,
    pdfpagemode=UseOutlines,
}

\usepackage{array}

\setlist[enumerate]{font=\bfseries}

% The first argument is the number of columns before the divider,
% and the second argument is the number of columns before the divider,
\newenvironment{abmatrix}[2]
{
    \left[
        \begin{array} {@{} *{#1}{c} | *{#2}c @{}}
}
{
        \end{array}
    \right]
}


% The argument is the number of variables in the system.
\newcolumntype{B}{>{{}}c<{{}}}
\newenvironment{system}[1]
{
    \begin{cases}
        \setlength{\arraycolsep}{0pt}
        \begin{array} {r @{}*{#1}{B r}}
}
{
        \end{array}
    \end{cases}
}

\DeclareMathOperator{\rank}{rank}
\DeclareMathOperator{\nullity}{nullity}
\DeclareMathOperator{\Span}{span}
\DeclareMathOperator{\tr}{tr}


\begin{document}

\setcounter{section}{2}
\section{Linear Transformations}

\subsection{Linear Transformations}
\begin{enumerate}[listparindent=\parindent]
\item[1.] Which of the following functions \(T\) from \(R^2\) into \(R^2\) are linear transformations?
    \begin{enumerate}[listparindent=\parindent]
        \item[(a)] \(T(x_1, x_2) = (1 + x_1, x_2)\);
            \[
                T((0, 0)) = (1, 0) \neq (0, 0)
            \]
            \(T\) is not a linear transformation.
        \item[(b)] \(T(x_1, x_2) = (x_2, x_1)\);
            \begin{gather*}
                T(c(x_1, x_2) + (y_1, y_2)) = \\
                T((cx_1 + y_1, cx_2 + y_2)) = \\
                (cx_2 + y_2, cx_1 + y_1)) = \\
                c(x_2 + x_1) + (y_2, + y_1)
            \end{gather*}
            \(T\) is a linear transformation.
        \item[(c)] \(T(x_1, x_2) = (x_1^2, x_2)\);
            \begin{gather*}
                T(cx_1 + y_1, cx_2 + y_2) = \\
                ((cx_1 + y_1)^2, cx_2 + y_2) = \\
                (c^2x_1^2 + 2cx_1y_1 + y_1^2, cx_2 + y_2) \neq \\
                c(x_1^2, x_2) + (y_1^2, y_2)
            \end{gather*}
            \(T\) is not a linear transformation.
        \item[(d)] \(T(x_1, x_2) = (\sin x_1, x_2)\);
            \begin{gather*}
                T(cx_1 + y_1, cx_2 + y_2) = \\
                (\sin(cx_1 + y_1), cx_2 + y_2) \neq \\
                c(\sin x_1, x_2) + (\sin y_1, y_2)
            \end{gather*}
            \(T\) is not a linear transformation.
        \item[(e)] \(T(x_1, x_2) = (x_1 - x_2, 0)\);
            \begin{gather*}
                T(c(x_1, x_2) + (y_1, y_2)) = \\
                (cx_1 + y_1 - cx_2 - y_2, 0) = \\
                (c(x_1 - x_2) + y_1 - y_2, 0)
            \end{gather*}
            \(T\) is a linear transformation.
    \end{enumerate}

\item[2.] Find the range, rank, null space, and nullity for the zero transformation and the identity transformation
    on a finite-dimensional space \(V\).

    For the zero transformation, the range is \(\{0\}\), the rank is 0, the null space is \(V\), and the nullity is \(\dim V\).
    For the identity transformation, the range is \(V\), the rank is \(\dim V\), the null space is \(\{0\}\), and the nullity is 0.

\item[4.] Is there a linear transformation \(T\) from \(R^3\) into \(R^2\) such that \(T(1, -1, 1) = (1, 0)\) and \(T(1, 1, 1) = (0, 1)\)?

    \((1, -1, 1)\) and \((1, 1, 1)\) are linearly independent, so there exists a basis of \(R^3\) that contains the two vectors.
    Then by Theorem 1, there exists a linear transformation that maps this basis into the desired outputs.

\item[5.] If
    \begin{gather*}
        \alpha_1 = (1, -1),\quad \beta_1 = (1, 0) \\
        \alpha_2 = (2, -1),\quad \beta_2 = (0, 1) \\
        \alpha_3 = (-3, 2),\quad \beta_3 = (1, 1)
    \end{gather*}
    is there a linear transformation \(T\) from \(\mathbb{R}^2\) into \(\mathbb{R}^2\) such that \(T\alpha_1 = \beta_i\) for \(i = 1, 2, 3\)?

    If such \(T\) exists, then \(T\alpha_3 = -(T\alpha_1 + T\alpha_2)\),
    but \(-(T\alpha_1 + T\alpha_2) = -(\beta_1 + \beta_2) = -\beta_3\) so \(T\alpha_3 \neq \beta_3\).
    Therefore there is no linear transformation with the given properties.

\item[6.] Describe explicitly (as in Exercises 1 and 2) the linear transformation \(T\) from \(F^2\) into \(F^2\)
    such that \(T\epsilon_1 = (a, b)\) and \(T\epsilon_2 = (c, d)\).

    \[T(x_1, x_2) = (ax_1 + cx_2, bx_1 + dx_2)\]

\item[8.] Describe explicitly a linear transformation from \(\mathbb{R}^3\) into \(\mathbb{R}^3\)
    which has its range the subspace spanned by \((1, 0, -1)\) and \((1, 2, 2)\).

    Let \(T\epsilon_1 = (1, 0, -1), T\epsilon_2 = (1, 2, 2), T\epsilon_3 = (0, 0, 0)\), then
    \[T(x_1, x_2, x_3) = (x_1 + x_2, 2x_2, -x_1 + 2x_2) = x_1(1, 0, -1) + x_2(1, 2, 2)\]

\item[9.] Let \(V\) the vector space of all \(n \times n\) matrices over the field \(F\), and let \(B\) be a fixed \(n \times n\) matrix.
    If \(T(A) = AB - BA\) verify that \(T\) is a linear transformation from \(V\) to \(V\).

    If \(X\) and \(Y\) are some \(n \times n\) matrices.
    \begin{gather*}
        T(cX + Y) = (cX + Y)B - B(cX + Y) \\
        T(cX + Y) = cXB + YB - cBX - BY \\
        T(cX + Y) = cXB - cBX + YB - BY \\
        T(cX + Y) = cT(X) + cT(Y)
    \end{gather*}

\item[10.] Let \(V\) the set of all complex numbers regarded as a vector space over the field of \textit{real} numbers (usual operations).
    Find a function from \(V\) to \(V\) which is a linear transformation on the above vector space, but which is not a linear transformation on \(\mathbb{C}^1\),
    i.e., which is not complex linear.

\item[11.] Let \(V\) be the space of \(n \times 1\) matrices over \(F\) and let \(W\) be the space of \(m \times 1 \) matrices over \(F\).
    Let \(A\) be a fixed \(m \times m\) over \(F\) and let \(T\) be the linear transformation from \(V\) to \(W\) defined by \(T(X) = AX\).
    Prove that \(T\) is the zero transformation if and only if \(A\) is the zero matrix.

\item[12.] Let \(V\) be an \(n\)-dimensional vector space over the field \(F\) and let \(T\) be a linear transformation from \(V\) into \(V\)
    such that the range and null space of \(T\) are identical. Prove that \(n\) is even. (Can you give an example of such a linear transformation \(T\)?)

\end{enumerate}

\end{document}
