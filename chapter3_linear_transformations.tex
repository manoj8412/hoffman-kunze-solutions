\documentclass{article}
\usepackage{array}
\usepackage{amssymb}
\usepackage{amsmath}
\usepackage{enumitem}
\usepackage{hyperref}
\hypersetup{
    bookmarksnumbered=true,
    pdfpagemode=UseOutlines,
}

\usepackage{array}

\setlist[enumerate]{font=\bfseries}

% The first argument is the number of columns before the divider,
% and the second argument is the number of columns before the divider,
\newenvironment{abmatrix}[2]
{
    \left[
        \begin{array} {@{} *{#1}{c} | *{#2}c @{}}
}
{
        \end{array}
    \right]
}


% The argument is the number of variables in the system.
\newcolumntype{B}{>{{}}c<{{}}}
\newenvironment{system}[1]
{
    \begin{cases}
        \setlength{\arraycolsep}{0pt}
        \begin{array} {r @{}*{#1}{B r}}
}
{
        \end{array}
    \end{cases}
}

\DeclareMathOperator{\rank}{rank}
\DeclareMathOperator{\nullity}{nullity}
\DeclareMathOperator{\Span}{span}
\DeclareMathOperator{\tr}{tr}


\begin{document}

\setcounter{section}{2}
\section{Linear Transformations}

\subsection{Linear Transformations}
\begin{enumerate}[listparindent=\parindent]
\item[1.] Which of the following functions \(T\) from \(R^2\) into \(R^2\) are linear transformations?
    \begin{enumerate}[listparindent=\parindent]
        \item[(a)] \(T(x_1, x_2) = (1 + x_1, x_2)\);
            \[ T((0, 0)) = (1, 0) \neq (0, 0)\]
            \(T\) is not a linear transformation.
        \item[(b)] \(T(x_1, x_2) = (x_2, x_1)\);
            \begin{gather*}
                T(c(x_1, x_2) + (y_1, y_2)) = \\
                T((cx_1 + y_1, cx_2 + y_2)) = \\
                (cx_2 + y_2, cx_1 + y_1)) = \\
                c(x_2 + x_1) + (y_2, + y_1)
            \end{gather*}
            \(T\) is a linear transformation.
        \item[(c)] \(T(x_1, x_2) = (x_1^2, x_2)\);
            \begin{gather*}
                T(cx_1 + y_1, cx_2 + y_2) = \\
                ((cx_1 + y_1)^2, cx_2 + y_2) = \\
                (c^2x_1^2 + 2cx_1y_1 + y_1^2, cx_2 + y_2) \neq \\
                c(x_1^2, x_2) + (y_1^2, y_2)
            \end{gather*}
            \(T\) is not a linear transformation.
        \item[(d)] \(T(x_1, x_2) = (\sin x_1, x_2)\);
            \begin{gather*}
                T(cx_1 + y_1, cx_2 + y_2) = \\
                (\sin(cx_1 + y_1), cx_2 + y_2) \neq \\
                c(\sin x_1, x_2) + (\sin y_1, y_2)
            \end{gather*}
            \(T\) is not a linear transformation.
        \item[(e)] \(T(x_1, x_2) = (x_1 - x_2, 0)\);
            \begin{gather*}
                T(c(x_1, x_2) + (y_1, y_2)) = \\
                (cx_1 + y_1 - cx_2 - y_2, 0) = \\
                (c(x_1 - x_2) + y_1 - y_2, 0)
            \end{gather*}
            \(T\) is a linear transformation.
    \end{enumerate}

\item[2.] Find the range, rank, null space, and nullity for the zero transformation and the identity transformation
    on a finite-dimensional space \(V\).

    For the zero transformation, the range is \(\{0\}\), the rank is 0, the null space is \(V\), and the nullity is \(\dim V\).
    For the identity transformation, the range is \(V\), the rank is \(\dim V\), the null space is \(\{0\}\), and the nullity is 0.

\item[4.] Is there a linear transformation \(T\) from \(R^3\) into \(R^2\) such that \(T(1, -1, 1) = (1, 0)\) and \(T(1, 1, 1) = (0, 1)\)?

    \((1, -1, 1)\) and \((1, 1, 1)\) are linearly independent, so there exists a basis of \(R^3\) that contains the two vectors.
    Then by Theorem 1, there exists a linear transformation that maps this basis into the desired outputs.

\item[5.] If
    \begin{gather*}
        \alpha_1 = (1, -1),\quad \beta_1 = (1, 0) \\
        \alpha_2 = (2, -1),\quad \beta_2 = (0, 1) \\
        \alpha_3 = (-3, 2),\quad \beta_3 = (1, 1)
    \end{gather*}
    is there a linear transformation \(T\) from \(\mathbb{R}^2\) into \(\mathbb{R}^2\) such that \(T\alpha_1 = \beta_i\) for \(i = 1, 2, 3\)?

    If such \(T\) exists, then \(T\alpha_3 = -(T\alpha_1 + T\alpha_2)\),
    but \(-(T\alpha_1 + T\alpha_2) = -(\beta_1 + \beta_2) = -\beta_3\) so \(T\alpha_3 \neq \beta_3\).
    Therefore there is no linear transformation with the given properties.

\item[6.] Describe explicitly (as in Exercises 1 and 2) the linear transformation \(T\) from \(F^2\) into \(F^2\)
    such that \(T\epsilon_1 = (a, b)\) and \(T\epsilon_2 = (c, d)\).

    \[T(x_1, x_2) = (ax_1 + cx_2, bx_1 + dx_2)\]

\item[8.] Describe explicitly a linear transformation from \(\mathbb{R}^3\) into \(\mathbb{R}^3\)
    which has its range the subspace spanned by \((1, 0, -1)\) and \((1, 2, 2)\).

    Let \(T\epsilon_1 = (1, 0, -1), T\epsilon_2 = (1, 2, 2), T\epsilon_3 = (0, 0, 0)\), then
    \[T(x_1, x_2, x_3) = (x_1 + x_2, 2x_2, -x_1 + 2x_2) = x_1(1, 0, -1) + x_2(1, 2, 2)\]

\item[9.] Let \(V\) the vector space of all \(n \times n\) matrices over the field \(F\), and let \(B\) be a fixed \(n \times n\) matrix.
    If \(T(A) = AB - BA\) verify that \(T\) is a linear transformation from \(V\) to \(V\).

    If \(X\) and \(Y\) are some \(n \times n\) matrices.
    \begin{gather*}
        T(cX + Y) = (cX + Y)B - B(cX + Y) \\
        T(cX + Y) = cXB + YB - cBX - BY \\
        T(cX + Y) = cXB - cBX + YB - BY \\
        T(cX + Y) = cT(X) + cT(Y)
    \end{gather*}

\item[10.] Let \(V\) the set of all complex numbers regarded as a vector space over the field of \textit{real} numbers (usual operations).
    Find a function from \(V\) to \(V\) which is a linear transformation on the above vector space, but which is not a linear transformation on \(\mathbb{C}^1\),
    i.e., which is not complex linear.

    Let \(T(x + yi) = x - yi\). If \(c\) is real, then \[(T(c(x + yi)) = cx - cyi = c(x - yi) = cT(x + yi),\]
    but \[T(i(i)) = T(-1) = -1 \neq 1 = i(-i) = iT(i)\]

\item[11.] Let \(V\) be the space of \(n \times 1\) matrices over \(F\) and let \(W\) be the space of \(m \times 1 \) matrices over \(F\).
    Let \(A\) be a fixed \(m \times n\) over \(F\) and let \(T\) be the linear transformation from \(V\) to \(W\) defined by \(T(X) = AX\).
    Prove that \(T\) is the zero transformation if and only if \(A\) is the zero matrix.

    \(AX\) is some linear combination of the column vectors of \(A\); \(AX = \alpha_1x_1 + \dots \alpha_nx_n\) where \(\alpha_i\) are columns of \(A\).
    This is only equal to the zero vector for any arbitrary \(X\) if and only if \(A = 0\).

\item[12.] Let \(V\) be an \(n\)-dimensional vector space over the field \(F\) and let \(T\) be a linear transformation from \(V\) into \(V\)
    such that the range and null space of \(T\) are identical. Prove that \(n\) is even. (Can you give an example of such a linear transformation \(T\)?)

    If the range and the null space of \(T\) are identical, then their dimensions are the same.
    \begin{gather*}
        \rank T + \nullity T = \dim V \\
        \rank T + \rank T = n \\
        2(\rank T) = n
    \end{gather*}
    Therefore \(n\) is even.

    For the space of \(2 \times 1\) matrices, the linear transformation \[T(A) = \begin{bmatrix}0 & 1 \\ 0 & 0\end{bmatrix}A\] is one such transformation.
    The range of \(T\) is spanned by \((1, 0)\), and \(AX = 0\) if and only if \(X\) is in the form \(\begin{bmatrix} a \\ 0 \end{bmatrix}\) so the null space is spanned by \((1, 0)\) as well.

\item[13.] Let \(V\) be a vector space and \(T\) a linear transformation from \(V\) into \(V\). Prove that the two following statements about \(T\) are equivalent.
    \begin{enumerate}
        \item[(a)] The intersection of the range of \(T\) and the null space of \(T\) is the zero subspace of \(V\).
        \item[(b)] If \(T(T\alpha) = 0\), then \(T\alpha = 0\).
    \end{enumerate}

    Suppose statement (a) is true. If \(T(T\alpha) = 0\), it must be the case that \(T\alpha\) is in the null space of \(T\).
    Since it is also clearly in its range, \(T\alpha\) must be 0.

    Suppose statement (b) is true; \(T\alpha\) is in the range and the null space of \(T\), and \(T\alpha = 0\).
    The zero vector is the only vector that is both in the range and the null space, so the intersectino of the range and null space is the zero subspace.
\end{enumerate}

\subsection{The Algebra of Linear Transformations}
\begin{enumerate}[listparindent=\parindent]
\item[1.] Let \(T\) and \(U\) be the linear operations on \(\mathbb{R}^2\) defined by
    \[ T(x_1, x_2) = (x_2, x_1) \text{ and } U(x_1, x_2) = (x_1, 0). \]

    \begin{enumerate}[listparindent=\parindent]
        \item[(a)] How would you describe \(T\) and \(U\) geometrically?

            \(T\) is a reflection across the line \(y = x\), and \(U\) is projection onto the x-axis.

        \item[(b)] Give rules like the ones defining \(T\) and \(U\) for each of the transformations
            \((U + T), UT, TU, T^2, U^2\).

            \[(U + T)(x_1, x_2) = (x_1 + x_2, x_1)\]
            \[UT(x_1, x_2) = (x_2, 0)\]
            \[TU(x_1, x_2) = (0_1, x_1)\]
            \[T^2(x_1, x_2) = (x_1, x_2)\]
            \[U^2(x_1, x_2) = (x_1, 0)\]
    \end{enumerate}

\item[2.] Let \(T\) be the (unique) linear operator on \(\mathbb{C}^3\) for which
    \[ T\epsilon_1 = (1, 0, i),\quad T\epsilon_2 = (0, 1, 1),\quad T\epsilon_3 = (i, 1, 0)\]
    Is \(T\) invertible?

    The matrix
    \[
        \begin{bmatrix}
            1 & 0 & i \\
            0 & 1 & 1 \\
            i & 1 & 0
        \end{bmatrix}
    \]
    is not invertible so \(\{T\epsilon_1, T\epsilon_2, T\epsilon_3\}\) are linearly dependent and cannot be a basis of \(\mathbb{C}^3\).
    Therefore \(T\) is not invertible.

\item[3.] Let \(T\) be the linear operator on \(\mathbb{R}^3\) defined by
    \[T(x_1, x_2, x_3) = (3x_1, x_1 - x_2, 2x_1 + x_2 + x_3).\]
    Is \(T\) invertible? If so, find a rule for \(T^{-1}\) like the one which defines \(T\).

    If \(T(x_1, x_2, x_3) = 0\), then
    \[
        \begin{system}{3}
            3x_1 && && &=& 0 \\
            x_1 &-& x_2 && &=& 0 \\
            2x_1 &+& x_2 &+& x_3 &=& 0 \\
        \end{system}
    \]
    \(x_1 = x_2 = x_3\), so \(T\) is non-singular and therefore invertible.

    \[
        \begin{system}{3}
            3y_1 && && &=& x_1 \\
            y_1 &-& y_2 && &=& x_2 \\
            2y_1 &+& y_2 &+& y_3 &=& x_3 \\
        \end{system}
    \]

    Solve this system to find
    \[ T^{-1}(x_1, x_2, x_3) = (\frac{x_1}{3}, \frac{x_1}{3} - 2, -x_1 + x_2 + x_3) \]

\item[5.] Let \(\mathbb{C}^{2 \times 2}\) be the complex vector space of \(2 \times 2\) matrices with complex entries.
    Let \[ B = \begin{bmatrix} 1 & -1 \\ -4 & 4 \end{bmatrix} \] and let \(T\) be the linear operator on \(\mathbb{C}^{2 \times 2}\)
    defined by \(T(A) = BA\). What is the rank of \(T\)? Can you describe \(T^2\)?

    \(B\bigl[\begin{smallmatrix} a & c \\ b & d \end{smallmatrix}\bigr] = 0\) implies \(a = b\) and \(c = d\), so the null space of \(T\) is spanned by two vectors:
    \begin{gather*}
        B\begin{bmatrix} 1 & 0 \\ 1 & 0 \end{bmatrix} = \begin{bmatrix} 0 & 0 \\ 0 & 0 \end{bmatrix} \\
        B\begin{bmatrix} 0 & 1 \\ 0 & 1 \end{bmatrix} = \begin{bmatrix} 0 & 0 \\ 0 & 0 \end{bmatrix}
    \end{gather*}
    Therefore \(\nullity(B)= 2\). Since \(\dim \mathbb{C}^{2 \times 2} = 4\), \(\rank(B)= 2\).

    \[ T^2(A) = BBA = \begin{bmatrix} 5 & -5 \\ -20 & 20 \end{bmatrix}A \]

\item[6.] Let \(T\) be a linear transformation from \(\mathbb{R}^3\) into \(\mathbb{R}^2\),
    and let \(U\) be a linear transformation from \(\mathbb{R}^2\) into \(\mathbb{R}^3\).
    Prove that the transformation \(UT\) is not invertible. Generalize the theorem.

    The range of \(T\) is a subspace of \(\mathbb{R}^2\), implying \(\rank(T) \leq 2\).
    \begin{gather*}
        \nullity(T) = 3 - \rank(T) \\
        \nullity(T) \geq 1
    \end{gather*}

    The inverse of \(UT\) is \(T^{-1}U^{-1}\), but \(T^{-1}\) does not exist and \(UT\) is not invertible.

    To generalize, let \(T\) be a linear transformation from \(V\) into \(W\) and
    let \(S\) be a linear transformation from \(W\) into \(V\).
    where \(V\) and \(W\) are vector spaces with \(\dim V > \dim W\).
    Then \(TS\) is not invertible.

\item[7.] Find two linear operators \(T\) and \(U\) on \(\mathbb{R}^2\) such that \(TU = 0\) but \(UT \neq 0\).

    \(T\) must not be invertible, as the null space of \(T\) is the range of \(U\) implying \(U = UT = 0\).
    Similarly, \(U\) must not be invertible as it implies the null space of \(T\) is \(\mathbb{R}^2\) and \(T = UT = 0\).

    \[T(x, y) = (x - y, x - y), \quad U(x, y) = (x + y, x + y)\]

\item[8.] Let \(V\) be a vector space over the field \(F\) and \(T\) a linear operator on \(V\).
    If \(T^2 = 0\), what can you say about the relation of the range of \(T\) to the null space of \(T\)?
    Give an example of a linear operator \(T\) such that \(T^2 = 0\) but \(T \neq 0\).

    The range of \(T\) is its null space.
    \[ T(x, y) = (0, x) \]

\item[9.] Let \(T\) be a linear operator on the finite-dimensional space \(V\).
    Suppose there is a linear operator \(U\) on \(V\) such that \(TU = I\).
    Prove that \(T\) is invertible and \(U = T^{-1}\).
    Give an example which shows that this is false when \(V\) is not finite-dimensional.
    (\textit{Hint:} Let \(T = D\), the differentiation operator on the space of polynomials.)

    If \(TU = I\), the null space of \(TU\) is the zero subspace since \(TUx = 0\) if and only if \(x = 0\).
    Therefore by Theorem 9, \(TU\) has the range \(V\) and is invertible.
    The range of \(T\) must also be \(V\), so it is invertible as well. Then
    \begin{gather*}
        TU = I \\
        T^{-1}TU = T^{-1} \\
        U = T^{-1}
    \end{gather*}

    Let \(V\) be the space of polynomials, \(D\) the differentiation operator, and \(E\) the indefinite integral operator.
    \(DE = I\), but \(ED \neq I\).

\end{enumerate}

\end{document}
